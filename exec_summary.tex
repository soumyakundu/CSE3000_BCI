\documentclass[12pt]{article}
\usepackage{amsmath}
\usepackage[top=1in, bottom=1in, left=1in, right=1in]{geometry}
\usepackage{multicol}
\usepackage{wrapfig}
\usepackage{listings}
%\usepackage{enumerate}
\usepackage{enumitem}
\usepackage[hyphens]{url}
\usepackage{hyperref}
\setlength{\columnsep}{0.1pc}

\title{\textbf{Brain-Computer Interface (BCI)}}
\author{Soumya Kundu and Christopher Oldham \vspace{0.1in} \\ soumya.kundu@uconn.edu, christopher.oldham@uconn.edu}
\date{}

\begin{document}

\maketitle
\vspace{-0.3in}

\section*{Introduction}

A brain-computer interface (BCI) is a device that provides the brain with a new, non-muscular communication and control channel [3]. The goal of a BCI is to provide a command signal directly form the brain, as that command can serve as a new functional output to control disabled body parts or physical devices, such as computers or robotic limbs [4]. Therefore, such a technology has tremendous potential to help those whose injuries have left them with a functioning central nervous system that can no longer control the actions of the peripheral nervous system [1].

However, there are numerous ethical concerns regarding the rapid development of BCI technologies. Implants are already being developed to replace damaged parts of the brain: two-way communication between brain and computer, bypassing the senses, is already becoming a reality. As the line between ourselves and our devices is blurred, we believe this poses unprecedented ethical and security questions. Do we maintain agency if part of our thinking is computer-regulated? Might humans become hackable?

For the first time in the history of our species, we will have the ability to integrate technology directly into our physiology. While this ability can immensely improve our understanding of the human brain and the lives of those with nervous system disorders, the social implications of this technology in the long run are vastly unknown and should be a subject of great debate.

\section*{Required Reading}

\begin{enumerate}[leftmargin=0.6cm, label={[\arabic*]}]

\item Wolpe, P.R. (2007). Ethical and Social Challenges of Brain-Computer Interfaces. AMA Journal of Ethics, 9(2), 128-131. Retrieved from \url{http://journalofethics.ama-assn.org/2007/02/msoc1-0702.html}

\item Rosahl, S.K. (2007). Neuroprosthetics and Neuroenhancement: Can We Draw a Line? AMA Journal of Ethics, 9(2), 132-139. Retrieved from \url{http://journalofethics.ama-assn.org/2007/02/msoc2-0702.html}

\end{enumerate}

\section*{Optional Reading}

\begin{enumerate}[leftmargin=0.6cm, label={[\arabic*]}]

\setcounter{enumi}{2}

\item Wolpaw, J.R., Birbaumer, N., McFarland, D.J., Pfurtscheller, G., and Vaughan, T.M. (2002). Brain-computer interfaces for communication and control. Clin Neurophysiol, 113, 767--791. Retrieved from \url{http://www.sciencedirect.com/science/article/pii/S1388245702000573}

\item Donoghue, J.P. (2002). Connecting cortex to machines: recent advances in brain interfaces. Nat Neurosci. 5 (Suppl), 1085--1088. Retrieved from \url{http://www.nature.com/neuro/journal/v5/n11s/full/nn947.html}

\sloppy

\item Graimann, B., Allison, B., and Pfurtscheller, G. (eds.). (2010). “Brain–Computer Interfaces: A Gentle Introduction,” in Brain–Computer Interfaces, The Frontiers Collection (Berlin: Springer-Verlag), 1--27. Retrieved from \url{http://www.springer.com/cda/content/document/cda_downloaddocument/9783642020902-c1.pdf?SGWID=0-0-45-1015086-p173959822}

\end{enumerate}

\end{document}
