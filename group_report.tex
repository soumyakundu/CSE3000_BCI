\documentclass[12pt]{article}
\usepackage{amsmath}
\usepackage[top=1in, bottom=1in, left=1in, right=1in]{geometry}
\usepackage{multicol}
\usepackage{wrapfig}
\usepackage{listings}
%\usepackage{enumerate}
\usepackage{enumitem}
\usepackage[hyphens]{url}
\usepackage{hyperref}
\usepackage{setspace}
\setlength{\columnsep}{0.1pc}

\title{\textbf{Brain-Computer Interface (BCI)}}
\author{Soumya Kundu and Christopher Oldham \vspace{0.1in} \\ soumya.kundu@uconn.edu, christopher.oldham@uconn.edu}
\date{}

\begin{document}

\maketitle
\vspace{-0.3in}

\section*{Introduction}

\doublespacing

A brain-computer interface (BCI) is a device that provides the brain with a new, non-muscular communication and control channel [3]. The goal of a BCI is to provide a command signal directly form the brain, as that command can serve as a new functional output to control disabled body parts or physical devices, such as computers or robotic limbs [4]. Therefore, such a technology has tremendous potential to help those whose injuries have left them with a functioning central nervous system that can no longer control the actions of the peripheral nervous system [1].

However, there are numerous ethical concerns regarding the rapid development of BCI technologies. Implants are already being developed to replace damaged parts of the brain: two-way communication between brain and computer, bypassing the senses, is already becoming a reality. As the line between ourselves and our devices is blurred, we believe this poses unprecedented ethical and security questions. Do we maintain agency if part of our thinking is computer-regulated? Might humans become hackable?

For the first time in the history of our species, we will have the ability to integrate technology directly into our physiology. While this ability can immensely improve our understanding of the human brain and the lives of those with nervous system disorders, the social implications of this technology in the long run are vastly unknown and should be a subject of great debate.

\singlespacing

\section*{Required Reading}

\begin{enumerate}[leftmargin=0.6cm, label={[\arabic*]}]

\item Wolpe, P.R. (2007). Ethical and Social Challenges of Brain-Computer Interfaces. AMA Journal of Ethics, 9(2), 128-131. Retrieved from \url{http://journalofethics.ama-assn.org/2007/02/msoc1-0702.html}

\item Rosahl, S.K. (2007). Neuroprosthetics and Neuroenhancement: Can We Draw a Line? AMA Journal of Ethics, 9(2), 132-139. Retrieved from \url{http://journalofethics.ama-assn.org/2007/02/msoc2-0702.html}

\end{enumerate}

\section*{Optional Reading}

\begin{enumerate}[leftmargin=0.6cm, label={[\arabic*]}]

\setcounter{enumi}{2}

\item Wolpaw, J.R., Birbaumer, N., McFarland, D.J., Pfurtscheller, G., and Vaughan, T.M. (2002). Brain-computer interfaces for communication and control. Clin Neurophysiol, 113, 767--791. Retrieved from \url{http://www.sciencedirect.com/science/article/pii/S1388245702000573}

\item Donoghue, J.P. (2002). Connecting cortex to machines: recent advances in brain interfaces. Nat Neurosci. 5 (Suppl), 1085--1088. Retrieved from \url{http://www.nature.com/neuro/journal/v5/n11s/full/nn947.html}

\sloppy

\item Graimann, B., Allison, B., and Pfurtscheller, G. (eds.). (2010). “Brain–Computer Interfaces: A Gentle Introduction,” in Brain–Computer Interfaces, The Frontiers Collection (Berlin: Springer-Verlag), 1--27. Retrieved from \url{http://www.springer.com/cda/content/document/cda_downloaddocument/9783642020902-c1.pdf?SGWID=0-0-45-1015086-p173959822}

\end{enumerate}

\newpage

\section*{Group Report}

\doublespacing

The presentation of this topic led to our classmates raising numerous interesting points, and together we robustly disccussed all of them. By our observation, our classmates fell with roughly equal numbers into two categories: those who had specifically heard of BCIs and those who had not. A product of our more esoteric topic selection, this range of topic familiarity in the audience meant our discussions had a healthy mix of premeditated ideas and on-the-fly analysis.

We started the presentation with a short video clip of Elon Musk, founder of SpaceX and Tesla and a notorious innovator and futurist. The clip was cut from a longer interview where Musk outlined his biggest concerns for the future, but in our clip Musk explained that he believes a top goal for the current generation of engineers, researchers, and scientists should be to increase the bandwidth between our brain and the computer-based devices that we use. This led us to the introduction of our topic of BCIs as one example of technology that can accomplish this goal of seamless communication between humans and computers.

First, we talked about current applications of BCI technology. We used a short video clip featuring theoretical physicist Dr. Michio Kaku to introduce the concept of using BCIs to help parapalegics. We asked the audience if they had ever experienced sleep-paralysis,  and most members indicated that they had. This allowed us to make a strong connection to those suffering from locked-in syndrome (LIS): a state of living similar to perpetual sleep-paralysis where the patient is conscious but retains little to no control of their muscles. Sufferers of LIS are one of the major benefactors of current BCI technology. This allowed us to make a strong case for BCIs as they are now: this technology can allow such patients to take advantage of communication and control channels with computers and allow them to perform simple tasks. BCI technology provides these patients with a little bit of autonomy in a life where they are mostly left at the mercy of their caregivers.

We asked the audience if they had heard of current BCI technologies or any issues therewith, and students recalled mouse-moving applications and a drone-control project done by a UConn student. When asked about issues with the current technology, most students were eager to extrapolate to issues with future technology, a topic we discussed in detail later in the presentation. We corroborated the suggestions given by the audience, and also included applications such as wheelchair control and mecanical arm articulation. Just as the audience struggled, we too struggled to find any serious ethical issues with the rather underdeveloped current technology, however we did note its limitations and drawbacks, and touched on the issue of consent with neurologically impaired patients.

When asked about possible future applications and concerns, the audience started by extrapolating the technologies they had seen to already exist. At one point however, a student introduced the idea of two-way BCI communication and Brain-to-Brain Interfacing, and many other students caught this buzz. A rich foliage of hypothetical leaves and philosophical flowers flourished from this branch of the conversation. In terms of concerns, there was a large interest (that we shared) regarding security, however there was also more concern with reliability and overdependence than we anticipated. As one student quipped: ``what if the computer in your brain was a Windows?" We discussed some near-future applications such as exoskeleton control and basic brain prosthesis, but quickly graduated to the more interesting and speculative topic of Brain-to-Brain interfacing. One student was adamant that such technology was mere science fiction, and so we were pleased to show footage of a recent experiment where two mice communicated telepathically using a Brain-to-Brain interface.

Our topic was considerably more esoteric than most, nevertheless we were very pleased with the rich discussion and novel ideas broached when the audience applied their thinking to a topic they were unfamiliar with. Surely a main goal of this course is to prepare students for the ethical analysis of technologies that do not yet exist, technologies that they will be involved in developing in the future. It was therefore reassuring to see such broad and nuanced discussion from our peers despite their lack of familiarity with Brain Computer Interfaces.



\end{document}
